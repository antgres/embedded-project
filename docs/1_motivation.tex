\section{Motivation}\label{motivation}

The motivation behind this project was to create a system that can take over some gardening work to reduce the time spent on checking on the plants or watering them. The idea is to implement sensors in and around the garden bed to track the surroundings of the plant. \\

Then, with the data from the sensors, it is easier for the user to know when to water the plants or do similar work on the plants. An addition to the sensor is an actuator which can take over simple work on the plant. Watering the plant or controlling the light of an ultraviolet lamp could be some of the tasks that the actuator could take on. And through the collection of data over a longer period of time, the tasks could be done with greater precision. \\

Even when going on a holiday, it would be possible for the user to control this system considering it is connected to the internet. And this is the key motivation. A system to take care of plants or similar objects remotely. \\

%How to realise the vision mentioned above is introduced int he se the scope of this project.




%The motivation behind this project was to have an automated home gardening system so that the user can check on his plants while not being at home and for example water the plants if needed. \\



%This would require to place a sensor in the garden bed that for example measures the moisture of the soil and an actuator which then waters the plants depending on the moisture. \\


\section{Scope of this project}\label{scope}

The goal of this project is to create a frame with which it is possible to have a home gardening system. This means that the whole system has to be modular so that the user can decide which sensors and actuators are needed for the certain situation. \\

The frame consists of three parts which are the webpage or frontend that serves as the human-machine-interface. Then there is the backend that contains the database and the interface for the communication between the frontend to the microcontroller and the backend. And lastly the third part is the microcontroller that processes the data coming from the sensor and actuator. \\

These parts have to work seamingless together ensuring a greater experience when using the system. 

