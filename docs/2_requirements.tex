\section{Requirements}\label{req}
This chapter contains the requirements which have been set for this project. The requirements are divided into requirements for frontend, backend and for the microcontroller. These also consider the experience of the project participants for each topic. \\ 

\subsection{Frontend} \label{reqfront}
Considering the requirements for the frontend there are two goals that have to be fullfilled. These are to let the user see all actuator-sensor-pairs (AS-Pairs) in the system and to add any new AS-Pair to the system. \\

A webpage containing a user interface (UI) would be the best choice for these goals. This enables the user to have access on the system from any device and from any place considering that it is connect to the internet. \\

The UI should also be kept simple to ensure a great user experience. The registered AS-Pairs have to stand out and they should be the focus of the webpage. Adding a pair has to be intuitive and self-explanatory. Meaning that it is set what information the user has to add to be able to add an AS-Pair. \\

Further there has to be made a connection between the frontend and the backend. To enable the above mentioned requirements. \\ 

\subsection{Backend and AS-Pair} 

For the subsystem \textit{backend and AS-Pair}, the following requirements are defined:
\begin{itemize}
    \item The subsystem shall be scalable, expandable and maintainable.
    \item For the first implementation, an MQTT interface is to be used.
    \item The AS-Pair should be visualised and controlled by a frontend.
\end{itemize}